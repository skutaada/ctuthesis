% arara: pdflatex: { synctex: yes }
% arara: makeindex: { style: ctuthesis }
% arara: bibtex

% The class takes all the key=value arguments that \ctusetup does,
% and a couple more: draft and oneside
\documentclass[twoside]{ctuthesis}

\ctusetup{
	preprint = \ctuverlog,
	mainlanguage = english,
	titlelanguage = english,
%	mainlanguage = czech,
	otherlanguages = {slovak,english},
	title-czech = {Moje bakalářka se strašně, ale hrozně dlouhým předlouhým názvem},
	title-english = {Drone detection using neural networks from combined RGB camera and LiDAR data},
	subtitle-czech = {Cesta do tajů kdovíčeho},
%	subtitle-english = {Journey to the who-knows-what wondeland},
	doctype = B,
	faculty = F3,
	department-czech = {Katedra matematiky},
	department-english = {Department of Cybernetics},
	author = {Adam Škuta},
	supervisor = {Matouš Vrba},
	supervisor-address = {Ústav X, \\ Uliční 5, \\ Praha 99},
	supervisor-specialist = {Martin Saska},
	fieldofstudy-english = {Mathematical Engineering},
	subfieldofstudy-english = {Mathematical Modelling},
	fieldofstudy-czech = {Matematcké inženýrství},
	subfieldofstudy-czech = {Matematické modelování},
	keywords-czech = {slovo, klíč},
	keywords-english = {word, key},
	day = 10,
	month = 2,
	year = 2017,
	specification-file = {ctutest-zadani.pdf},
%	front-specification = true,
%	front-list-of-figures = false,
%	front-list-of-tables = false,
%	monochrome = true,
%	layout-short = true,
}

\ctuprocess

\addto\ctucaptionsczech{%
	\def\supervisorname{Vedoucí}%
	\def\subfieldofstudyname{Studijní program}%
}

\ctutemplateset{maketitle twocolumn default}{
	\begin{twocolumnfrontmatterpage}
		\ctutemplate{twocolumn.thanks}
		\ctutemplate{twocolumn.declaration}
		\ctutemplate{twocolumn.abstract.in.titlelanguage}
		\ctutemplate{twocolumn.abstract.in.secondlanguage}
		\ctutemplate{twocolumn.tableofcontents}
		\ctutemplate{twocolumn.listoffigures}
	\end{twocolumnfrontmatterpage}
}

% Theorem declarations, this is the reasonable default, anybody can do what they wish.
% If you prefer theorems in italics rather than slanted, use \theoremstyle{plainit}
\theoremstyle{plain}
\newtheorem{theorem}{Theorem}[chapter]
\newtheorem{corollary}[theorem]{Corollary}
\newtheorem{lemma}[theorem]{Lemma}
\newtheorem{proposition}[theorem]{Proposition}

\theoremstyle{definition}
\newtheorem{definition}[theorem]{Definition}
\newtheorem{example}[theorem]{Example}
\newtheorem{conjecture}[theorem]{Conjecture}

\theoremstyle{note}
\newtheorem*{remark*}{Remark}
\newtheorem{remark}[theorem]{Remark}

\setlength{\parskip}{5ex plus 0.2ex minus 0.2ex}

% Abstract in Czech
\begin{abstract-czech}
\end{abstract-czech}

% Abstract in English
\begin{abstract-english}
\end{abstract-english}

% Acknowledgements / Podekovani
\begin{thanks}
Děkuji ČVUT, že mi je tak dobrou \emph{alma mater}.
\end{thanks}

% Declaration / Prohlaseni
\begin{declaration}
Prohlašuji, že jsem předloženou práci vypracoval samostatně, a že jsem uvedl veškerou použitou literaturu.

V Praze, \ctufield{day}.~\monthinlanguage{title}~\ctufield{year}
\end{declaration}

% Only for testing purposes
\listfiles
\usepackage[pagewise]{lineno}
\usepackage{lipsum,blindtext}
\usepackage{mathrsfs} % provides \mathscr used in the ridiculous examples
\usepackage{todonotes}

\begin{document}

\maketitle

\part{Methodology}
\chapter{Sensors}
\section{Camera model}

\part{Task}
\chapter{Dataset}
The dataset for this work can be generated in two ways. The first is real-life drone shots mixed with point clouds from LiDAR mounted on top of a drone. The second is generating a dataset using a realistic virtual environment where a drone, camera and LiDAR are being emulated very close to their real-life counterparts. An advantage to this approach is that a great variety of environments can be chosen a lot of them often inaccessible otherwise (power plant, airport, snowy mountains out of season etc.). Therefore this approach will be chosen for the task.
\section{Unreal Engine}
Unreal Engine is a software tool used for creating realistic 3d environments, most often used as a video game engine.
\todo{citation https://www.unrealengine.com/en-US/features} 
It is written in C++ and open-source supporting a variety of pre-built environments and assets. For this work three diferent environments will be used for the creation of the dataset:
\begin{itemize}
	\item \todo[inline]{exact name. Snow}
	\item \todo[inline]{exact name. Park}
	\item \todo[inline]{exact name. City centre}
\end{itemize}
\todo{Pictures of the environments}
Together \todo[inline]{exact number of pictures} pictures and labels were generated using two drones. One drone was equipped with RGB camera and LiDAR sensor and was responsible for taking the pictures and pointclouds from LiDAR. The second one was used as a model for drone detection.
\section{AirSim}
Open-source plugin for Unreal Engine called AirSim was used for the generation of the dataset. It simulates realistic flight motions of drones as well as seven types of sensors \todo{airsim zdroj}, including RGB camera and LiDAR used for this task. AirSim supports both a C++ API as well as Python API, latter which was used for controlling the motion and capturing the dataset. Location of the second drone was generated through API call, which produces a location of the drone in global coordinate system of the map, which is later transformed to the local coordinates of the first drone carrying the LiDAR and RGB sensors.\todo{Transformacna matica?}. The capturing drone traveled on each map on a 3d cube grid:\todo[inline]{Grafika kocky po ktorej lietal dron}
\section{Camera Model}
 

\part{Your Party}

\blindmathtrue

\blinddocument

\begin{table}
\begin{ctucolortab}
\begin{tabular}{cc}
\bfseries Foo & \bfseries Bar \\\Midrule
foo1 & bar1 \\
foo2 & bar2
\end{tabular}
\end{ctucolortab}
\caption{Foobar.}
\label{tab:foobar}
\end{table}

\begin{figure}
\includegraphics[width=0.4\textwidth]{ctu_logo_black}
\caption{Black logo of the CTU in Pragueueue.}
\end{figure}

\begin{figure}[!t]
\includegraphics[width=0.4\textwidth]{ctu_logo_blue}
\caption{Blue logo of the CTU in Pragueueue.}
\end{figure}

\chapter{Conclusions}

\section{Test --- this is just a little test of something in the table of contents}

\subsection{Yes, table of contents}

\begin{theorem}\begin{enumerate} \item Bla \item Blo \end{enumerate} \end{theorem}


\medskip

\begin{proof}\begin{enumerate} \item[8] Bla \item Blo \end{enumerate} \end{proof}

\appendix

\printindex

\appendix

\bibliographystyle{amsalpha}
\bibliography{ctutest}

\ctutemplate{specification.as.chapter}

\end{document}